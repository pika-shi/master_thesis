\section{Related Work}
\label{sec:Related Work}

With explosive widespread use of microblogs, studies about
them have recently become frequently performed.  There are various
studies of microblogs, e.g., studies of the
classification microblogging messages from various point of
view\cite{irani2010study}, studies of ranking microblogging messages by
the content relevance and so on\cite{duan2010empirical}, or studies of focusing on
real-time nature of migroblogs\cite{takemura2012tweet,mathioudakis2010twittermonitor}.

In this study, we focus on the fact that Twitter is used for various
purposes. We attempt to classify Twitter users based on the
breadth of target scope of their information publishing, and apply this
classification scheme on Twitter search and so on.  We explain the
following previous studies: studies about the purpose of use of
Twitter in \ref{subsec:Purpose of Use}, studies of
classifying Twitter users and measuring their influence in
\ref{subsec:Twitter User}, studies useful for finding messages related
to only certain Twitter users in \ref{subsec:Some User}, and studies
about Twitter search in \ref{subsec:Twitter Search}.
We make the position of this study clear by introducing these studies.

\subsection{Purpose of Use of Twitter}
\label{subsec:Purpose of Use}

There has been many studies about the purpose of use of Twitter. Java
et al.\cite{java2007we} analyzed the topological and
geographical structure of Twitter's social network and attempted to
understand the user intentions and community structure in microblogging
services.  As a result, they found that the main types of user
intentions are daily chatter, conversation, sharing information and
reporting news.  Kwak et al.\cite{kwak2010twitter} reported that Twitter
is used both as a social network service and as a media for
disseminating or gathering information, and in its follower-following
topology analysis they have found a non-power-law follower distribution,
a short effective diameter, and low reciprocity, which all mark a
deviation from known characteristics of human social networks.  There
are many more studies about the purpose of use of Twitter\cite{wu2011says,zhao2009and}.

Ehrlich et al.\cite{ehrlich2010microblogging} conducted a
content analysis and examined the use of public microblogs
(Twitter) for public and private use by comparing internal
microblogs (in the workspace).  As a result, there were
significant differences in content.  The internal microblogs
were generally used to solicit technical assistance or as part of a
conservation, and the public microblogs were used for status
updates and to share general information.

In recent years, Twitter, one of public microblogs, is often
used for not only publishing information to the public but also having a
relationship to only a certain community.  It is able to be said that
this study focuses on the fact that we use Twitter for various purposes.

\subsection{Classification of Twitter Users and Measuring their Influence}
\label{subsec:Twitter User}

There are many studies focusing on Twitter users, e.g., studies which
classify them from various point of view, and studies which measure
their influence.

Studies focusing on the classification of Twitter users are performed
frequently and they have a wide variety of classification schemes, e.g.,
the classification based on their attributes such as political
orientation or ethnicity by leveraging observable information such as
the user behavior, network structure, and linguistic content of their
posting messages\cite{pennacchiotti2011machine}, the classification
into spam users or not by extracting observable features from the
collected candidate spam profiles, e.g., number of friends, text on the
profile, age, and so on\cite{lee2010social}, and the classification
into human users, bots, and cyborgs using entropy measures, machine
learning, and so on\cite{chu2010tweeting}.  Bots refer to automated
programs posting on Twitter, and cyborgs refer to either bot-assisted
humans or human-assisted bots, i.e., interweave characteristics of both
humans and bots.

The classification scheme proposed by Yan et
al.\cite{yan2013classifying} deeply relates to ours.  They proposed
methods to classify Twitter users into open accounts and closed
accounts.  An open account is the
account with a purpose for advertising or spreading information such as
a shop, a singer, a news agency, and so on.  On the other hand, a closed
account is the
account with a purpose for making friends or communication such as a
user publishing messages about daily log, feeling show, and so on.
This classification scheme is close to ours, but does not coincide with
ours because open accounts do not often publish information to the wide
public.  For example, a user publishing technical information
about programming to unspecific users is an open account though he is a
target user.

In microblogs like Twitter or other social network services
like Facebook, users correspond to nodes in social network graphs. As
well as the classification of users, i.e., nodes in the graphs, the
classification of edges, i.e., relationship between a user and his
followers, is related to our study. Leskovec et
al.\cite{leskovec2010predicting,leskovec2010signed} classified edges in
SNS into positive edges such as friendship, and negative edges such as
antagonism. Kunegis et al.\cite{kunegis2009slashdot} also use positive
edges and negative edges in Slashdot, a message board service, in order
to rank the users. Cheng et al.\cite{cheng2011predicting} and Hopcroft
et al.\cite{hopcroft2011will} studied the problem of predicting
reciprocity between two given Twitter users.

There are also studies focusing on measuring influence of Twitter
users.  Jianshu et al.\cite{weng2010twitterrank} focused on the problem of
identifying influential users of microblogs.  Cha et
al.\cite{cha2010measuring} analyzed the influence of them
by employing three measures that capture different perspectives:
indegree, retweets, and mentions.  Then they measured the dynamics of
influence across topic and time.  If target specificity of Twitter users
defined in this study is high, there is a high probability that they
have a big influence on Twitter, but how low target specificity of a
user is does not necessarily coincide with how big his influence is.

\subsection{Find Messages Related to Certain Twitter Users}
\label{subsec:Some User}

There are also studies useful for finding of microblogging messages
related to only a part of Twitter users.

Sakaki et al.\cite{sakaki2010earthquake} proposed a method of monitoring
messages in Twitter and detecting occurrences of a specific kind of event
in the real world, such as earthquakes or typhoons.  They produced a
probabilistic spatiotemporal model for the target event that can find
the center and the trajectory of the event location.  Ikawa et
al.\cite{ikawa2012location} attempted to discover the location where a
message was generated by using its textual content.  They learned
associations between a location and relevant keywords from past
messages, and guessed where a new message came from.  It is able to be
said that these studies are useful for finding messages in Twitter
related to certain geographical areas.

Sriram et al.\cite{sriram2010short} proposed approach which effectively
classifies the message to a predefined set of generic classes such as
News, Events, Opinions, Deals, and Private Messages.  They proposed to
use a small set of domain-specific features extracted from the user's
profile and text.  Nishida et al.\cite{nishida2011tweet} proposed a
method which uses data compression for classifying an unseen tweet as
being related to an interesting topic or not.  It is able to be
said that these studies are useful for finding messages to some specific
topics in Twitter.

As mentioned above, there are many studies useful for finding microblogs
related to only a part of Twitter users in various points of
view.  But these points of view exist in great number, so it is not an
effective approach to find these messages from each point of view.
Thus in this study, we attempted to measure Twitter users' target
specificity of information publishing in an integrated way.  In
addition, we roughly classified various causes of target specificity
into two categories.

\subsection{Twitter Search}
\label{subsec:Twitter Search}

The characteristic of search on microblogs is different
from that of Web search\cite{broder2002taxonomy} in that search on
microblogs can get information in real
time\cite{busch2012earlybird} and not only information published by the
mass media but also much casual information published by
individuals\cite{java2007we}.  Thus, a purpose of use of search on
microblogs often becomes a subject of study.

Teevan et al.\cite{teevan2011twittersearch} observed that people use
Twitter search to find temporally relevant information, e.g., breaking
news, real-time content, and popular trends, and information related to
people, e.g., content directed at the searcher, information about
people of interest, and general sentiment and opinion.  Furthermore,
they compared Twitter search with Web search and found that search
results on Twitter included more social chatter and social events, and
those on the Web included more basic fact and navigation content.
Massoudi et al.\cite{massoudi2011incorporating} proposed a retrieval
model for searching messages on microblogs for a given topic
of interest and a dynamic query expansion model for messages retrieval.
And Nagamoti et al.\cite{nagmoti2010ranking} described several
strategies for ranking messages of microblogs in a
real-time search engine.

As mentioned above, there are many studies about search on microblogs,
and contents of them are greatly various.  In this study, we
focus on the purpose of use of microblogs, and attempt to
apply it to Twitter Search.  It is able to be said that this study also
focuses on search on microblogs just like studies explained
above, but there has not been studied based on Twitter user's target
specificity of information publishing so far.