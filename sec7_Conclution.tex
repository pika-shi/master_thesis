\section{Conclusion}
\label{sec:Conclusion}

In this study, we focused on the fact that the wideness of target scope of
information publishing in Twitter varies greatly among users, and
proposed the method to classify Twitter
users from the point of view of how widely target scope of their
information publishing is, i.e., whether they publish information to the
public widely or publish information specified in certain users.

First, we defined target specificity of a user, as the
measure of to what degree target scope of his information publishing is
specified.  Second, based on this definition, we proposed the algorithm
computing a score of target specificity.  In this algorithm, we
focused on two parameters: (a) whether followers of a user are
consistent in a certain noticeable character or not, and (b) whether the
follower set of a user is covered with consistency subsets covering it
intermediately widely or not.
For computing the score, we proposed a couple of models computing
scores of consistency subsets: the
probabilistic model and the subtracting model, and a couple of
attributes for extracting consistency subsets from the follower set of a
user: common terms in profiles and location information and common
followees.
Then, we finally proposed four types of methods of classifying users
into target users and non target user based on the score.  We conducted
experiments to compare the performance of the methods, and the results
suggested that the method using a binary SVM classified users with the
highest accuracy.  The results also suggested that common terms in profiles
and location information were more useful for extracting consistency
subsets than common followees.

In addition, in this study, in regard to users classified into
``target specificity is high'' by the above method, we proposed the
the method of determining why their target
specificity is high.  We first analysed the causes of high target
specificity, and we classified users into three
categories based on them: (1) their target specificity is high because
they publish information specified for certain topics,
(2) because they publish information to the users specified extensionally, and
(3) in the cause of both (1) and (2).  Then, we constructed
a couple of approaches to classify them into three categories: a 3-class
classifier and 2 binary classifiers, based on various features of the
user which potentially correlate with each cause.  Our experimental
results suggested that a 2 binary SVM classified users with the highest
accuracy. The results also suggested that the two causes (1) and (2)
were highly independent of each other.

We believe that our proposed methods are applicable to some applications
e.g., Twitter search systems, and they will enrich users' experience on
Twitter.