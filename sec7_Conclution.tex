\section{Conclusion}
\label{sec:Conclusion}

In this study, we focus on the fact that the wideness of target scope of
information publishing varies greatly among users bacause Twitter is
used for various purpose, we proposed the method to classify Twitter
users from the point of view of how widely the target scope of their
information publishing is, i.e., whether they publish information to the
public widely or publish information specified in certain users.

First, we defined the target specificity of the Twitter user, as the
measure of how much target scope of their information publishing is
specified.  Second, based on this definition, we proposed the algorithm
of computing a score of the target specificity.  In this algorithm, we
focused on the two parameters: (a) whether followers of the user are
consistent in a certain noticeable character or not, and (b) whether the
follower set of the user is covered with consistency subsets which cover
intermediately widely or not.  For computing the score, we proposed a
couple of models computing a score of the consistency subset: the
probabilistic model and the subtracting model, and we compared the above
two models.  Furthermore, we proposed a couple of attributes for
extracting consistency subsets from the follower set of the user: common
terms in profiles and location information and common followees.  Then,
we finally proposed the method of classifying Twitter users based on the
score.  We conducted the experiment and our experimental results
confirmed that these two techniques improve the precision.

%実験結果

In addition, in this study, in regard to Twitter users classified into
``he target specificity is high”''y thehe above method, we proposed thy
the above method, we proposed the method of determining why their target
specificities are high.  We analysed the causes of high target
specificity, and based on them, we classified the user into three
categories: (1) they publish information specified for certain topics,
(2) they publish information to the users specified extensionally, and
(3) in the cause of both (1) and (2).  In this method, we constructed
3-class classifiers which classify a user into the above three
categories based on various features of the user which potentially
correlate with each cause, and we determined why the target specificity
is high by using them.
%実験結果