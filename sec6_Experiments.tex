\section{Experiments and Discussions}
\label{sec:Experiment}

In this chapter, we conduct experiments to evaluate our methods proposed
in \ref{sec:ClassificationMethod1} and \ref{sec:ClassificationMethod2},
and present the results obtained from them.  In addition, we discuss our
methods based on the results.

\subsection{Data Set}
\label{subsec:Data Set}

We collected the data set from the real Twitter data by using Twitter
API.

We first randomly selected 1,000 Twitter users whose timezone is Japan.
At this time, we omitted users who are followed from nobody and who post
no tweet in order to select only active users.  Then, we divided them in
two sets equally, i.e., each of which include 500 users.

Second, we had 6 experienced Twitter users as participants, all of whom
are male graduate students in engineering, from 23 to 25 years old.  We
assigned each set to 3 participants, and we asked each participant to
determine one of the following categories each user in the assigned set
is supposed to be in:

\begin{description}
\item[(i)] the user publishes information to the public widely,
\item[(ii)] the user publishes information specified for certain topics,
\item[(iii)] the user publishes information to the users specified
           extensionally, and
\item[(iv)] the user publishes information (ii) specified for certain
           topics (iii) to the users specified extensionally.
\end{description}

\noindent{These categories correspond to the category (b), (1), (2), (3)
in Figure~\ref{fig:Flow} respectively.}

Then, we selected users whose category at least 2 out of 3 participants
coincide with, and as a result, we were able to collect 93, 320, 375,
and 30 users in the category (i), (ii), (iii), and (iv).  We randomly
selected 90 users from the category (i), and 30 users from (ii), (iii),
and (iv) respectively.  We collected these 180 users in total, and we
used them as the data set.  Table~\ref{breakdown} shows the breakdown of
the data set: average and standard deviation of numbers of followers,
 followees, and tweets in each category.

\newcolumntype{I}{!{\vrule width 1.5pt}}
\newcommand{\bhline}[1]{\noalign{\hrule height #1}}
\begin{table}[t]
\caption{Average and standard deviation of numbers of followers,
 followees, and tweets in each category \label{breakdown}}
\begin{center}
\begin{tabular}{c|cIc|c|c|c}
\multicolumn{2}{cI}{} & \makebox[4em]{i} & \makebox[4em]{ii} &
 \makebox[4em]{iii} & \makebox[4em]{iv} \\ \bhline{1.5pt}
 \multirow{2}{*}{follower} & average & 475,679 & 58,142 & 573 & 82,942  \\
 & standard deviation & 535,894 & 171,784 & 1,389 & 262,161 \\ \hline
 \multirow{2}{*}{followee} & average & 11,274 & 3,353& 598 & 1,568 \\
 & standard deviation & 37,906 & 7,218 & 1,545 & 3,594 \\ \hline
 \multirow{2}{*}{tweet} & average & 9,763 & 9,992 & 8,829 & 5,677 \\
 & standard deviation & 14,607 & 23,572 & 29,505 & 6,600 \\
\end{tabular}
\end{center}
\end{table}

\begin{table}[t]
\caption{Average and standard deviation of numbers of followers,
 followees, and tweets in each category \label{ClassifierResultslog}}
\begin{center}
\begin{tabular}{cIc|cccc}
Removed Feature & \makebox[4em]{with all} & \makebox[3em]{i} &
 \makebox[3em]{ii} & \makebox[3em]{iii} & \makebox[3em]{iv} \\
 \bhline{1.5pt}
 3-class SVM & 67.8 & 64.4 & {\bf 72.2} & 67.8 & 65.5  \\ \hline
 2 binary SVMs & 68.9 & 65.6 & 71.1 & 67.8 & 67.8  \\ \hline
 3-class decision tree & 55.6 & 54.4 & 46.7 & 47.8 & 50.0 \\ \hline
 2 binary decision trees & 53.3 & 52.2 & 44.4 & 47.8 & 47.8  \\
\end{tabular}
\end{center}
\end{table}

\begin{table}[t]
\caption{Average and standard deviation of numbers of followers,
 followees, and tweets in each category \label{ClassificationDetailslog}}
\begin{center}
\begin{tabular}{c|cIc|cccc}
\multicolumn{2}{cI}{Removed Feature} & \makebox[4em]{with all} &
 \makebox[3em]{i} & \makebox[3em]{ii} &  \makebox[3em]{iii} &
 \makebox[3em]{iv} \\ \bhline{1.5pt}
 \multirow{2}{*}{SVM} & \makebox[4em]{topic} & 85.6 & 81.1 & {\bf 86.7}
                 & 84.4 & 84.4  \\
 & user & 83.3 & 84.4 & 84.4 & 83.3 & 82.2 \\ \hline
 \multirow{2}{*}{decision tree} & topic & 72.2 & 68.9 & 65.6 & 72.2 & 74.4 \\
 & user & 75.6 & 76.7 & 74.4 & 71.1 & 70.0 \\
\end{tabular}
\end{center}
\end{table}

Then, for each user, we collected at most 1,000 followers of the user,
and in regard to the followers who follow at most 1,000 users, we also
collected their followees.  We used them in order to evaluate our
methods.

\subsection{Experimental Settings and Libraries}
\label{subsec:Settings}

First, we conducted the experiments evaluating the method of classifying
Twitter users based on the target specificity of their information
publishing mentioned in \ref{}.  We used 90 users classified into the category (i) as target
users, and 90 users classified into the category (ii), (iii), and (iv)
as non target users.  We first computed
$\mathit{SpecificityScore}_{{\mathit{term}}}(u)$ and
$\mathit{SpecificityScore}_{{\mathit{followee}}}(u)$ for each user $u$,
and computed $\mathit{TargetSpecificity}(u)$ based on the above scores.
Then, we determined a threshold $\delta$ which can classifies target
users and non target users accurately the most, and evaluated the
classification results with $\delta$.  When computing
$\mathit{TargetSpecificity}(u)$, we used two models: the probabilistic
model and the subtracting model mentioned in \ref{subsubsec:Scoring},
and compared them.

Second, we conducted the experiments evaluating the method of determing
why the target specificity of the user is high in regard to the target
user.  We extracted users from the category (ii), (iii), and (iv) by 30
users, and used 90 users in total.  We first extracted features
mentioned in \ref{subsec:Classification Step2} from the user.  Then,
based on these features, we constructed 3-class classifiers which
classify users into three categories: (ii), (iii), and (iv), and
evaluated the classification results using 10-fold cross validation.  We
used two learning algorithms: SVM and the decision tree as classifiers,
and compared them.  For SVM, we used
LIBSVM\footnote{\url{http://www.csie.ntu.edu.tw/~cjlin/libsvm/}}, which
is a popular SVM library, with the Gaussian kernel.  For the decision
tree, we used scikit-learn
\footnote{\url{http://scikit-learn.org/stable/modules/tree.html}}.

We used twpro search API\footnote{\url{http://twpro.jp/doc/api/search}}
in order to get a number of users who have a certain term in their
profiles.  We also used
MeCab\footnote{\url{http://mecab.sourceforge.net/}} for morphological
analysis of Japanese sentences in profiles, local information, and
tweets of $u$.  Furthermore, we used
gensim\footnote{\url{http://radimrehurek.com/gensim/}} for using Latent
Dirichlet Allocation (LDA).

実験結果を図\ref{demography},図\ref{follow},図\ref{pr}に示す.
図\ref{demography}は,プロフィール・位置情報を用いて対象限定性を表した手法,図
\ref{follow}は,フォローの傾向を用いて対象限定性を表した手法を表しており,
それぞれスコアの高いものからソートしている.図\ref{pr}は,上記の二つの手
法で求めたスコアをそれぞれ降順にソートし,ターゲット型ユーザを正解
とした場合の,適合率・再現率曲線を表している.また,適合率・再現率曲線の
ベースラインとして,20件のユーザそれぞれのフォロワー数を昇
順にソートし,同じくターゲット型ユーザを正解としたものを用いている.

フォロワー内でのプロフィール・位置情報を用いた手法に関しては,図
\ref{demography}に示すように,ターゲット型ユーザの方が,非ターゲット
型ユーザよりも,対象限定性を表すスコアが高くなっており,これらを明確に分
類することができているといえる.また,図\ref{pr}に示すように,適合
率・再現率曲線も,ベースラインを上回っている.これらの結果より,プロフィー
ル・位置情報を用いて対象限定性を表す手法は,ターゲット型ユーザと非ターゲッ
ト型ユーザを分類するのに有効であるといえる.

フォロワー内でのフォローの傾向を用いた手法に関しても,図\ref{follow}に示
すように,ターゲット型ユーザの方が非ターゲット型ユーザよりも,対象限定性
を表すスコアが平均的に高くなった.しかし,これら二つの間の差は大きくなく,
適合率・再現率曲線に関しては,図\ref{pr}に示すように,ベースライン
を下回る結果となった.これらの原因としては,以下のようなものが考えられる.
\vspace{1ex}


{\bf(1)} 本実験では,フォロワーのフォローの傾向を用いた手法の中で,Twitter
       の日本人ユーザ全体でのフォローの頻度$\mathit{uf}$を求めているが,
       実際には,Twitterに登録しているものの,ほとんど利用していないユー
       ザも多く存在する.そのようなユーザの影響で,$\mathit{uf}$が必要以
       上に小さくなってしまっており,最終的なスコアに十分反映されていな
       いと考えられる.今後は,Twitterユーザ全体ではなく,Twitterのアクティブユーザ
       のみの中からフォローの頻度を求めることで,本問題の改善を図る予定
       である.\vspace{1ex}

{\bf(2)} フォロワー内に強く一貫した傾向がなくても,フォローに共通の傾
       向が見られる場合があると考えられる.本実験において,社会のニュー
       スを発信するユーザのフォロワーの多くが,社会のニュースを発信する
       別のユーザを同時にフォローしているというケースが見られたが,実際
       には,フォロワー内に社会のニュースに興味があるという弱い一貫性は
       あるものの,強く一貫した傾向は見られなかった.\vspace{1ex}

\subsection{実験結果の詳細と分析}
\label{sec:detail}

本節では,実験結果から,正解データとして用いたユーザの分析を行った結果を
示す.表\ref{detail}に,正解データ内のターゲット型ユーザ・非ターゲット型
ユーザのうち,各2ユーザの詳細を表している.各ユーザそれぞれに関して,
\ref{sec:Determine Specificity}節で定義した$\mathit{tf}$が高い3語と,
$\mathit{ff}$が高い3ユーザを示し,実験結果におけるそれらのスコアを示して
いる.

ターゲット型ユーザである@minnanomachi(写真に関するプロジェクトのア
カウント)に関しては,「写真」や「カメラ」といった語のスコアや,カメラ関連
のアカウントのスコアが高くなっており,そのアカウントを特徴付けるような結
果が得られているといえる.同様にターゲット型ユーザである@USJ\_BGM(USJに
関するアカウント)に関しても,「USJ」といった語のスコアや,USJ関連のアカ
ウントのスコアが高くなっている.また,「好き」といった一般に出現頻度の高
い語や,@masason(孫正義)といった広く一般に知られているユーザは,
$\mathit{tf}$や$\mathit{ff}$が高くなっているが,$\mathit{df}$や
$\mathit{uf}$によりそれらのスコアは下がっており,最終的なスコアは非常に小
さくなっていることが分かる.

非ターゲット型ユーザに関しては,「大好き」や「フォロー」といった,一般に
出現頻度の高い語や,@pamyurin(きゃりーぱみゅぱみゅ)や@TDR\_PR(東京
ディズニーリゾートに関するアカウント)といった,広く一般に知られているユー
ザが抽出されているが,$\mathit{df}$や$\mathit{uf}$によりこれらのスコアは
下がっている.特に,プロフィール・位置情報を用いた手法のスコアは全て0となって
いる.

このように,これらの例に関しては,フォロワー集合の中で特徴のある語やフォ
ローの傾向を抽出することができており,提案手法が有効であるといえる.ただ
し,現在の手法では,「写真」と「カメラ」というような同じトピックに関する
語を,完全に独立なものとして扱っている.このような語を関連付けて考え,対
象限定性を求める有効性を向上させることが今後の課題である.

\subsection{アプリケーションの実装}
\label{sec:Application}

本節では,本研究の提案手法を応用した,実装予定のアプリケーションについて
述べる.

本アプリケーションは,提案手法に基づき,Twitterの検索結果を3つのタブに分
類することで,ユーザのTwitter検索を支援するものである.以下に,表示する3
つのタブついてそれぞれ示している.\vspace{1ex}

{\bf (1)} 対象範囲が広いユーザの記事を表示する.ある内容に関する一般的な
ニュースのように,広く一般のユーザが興味を示すような記事を得ることができ
る.\vspace{1ex}

{\bf (2)} あるトピックに特化された情報を発信するユーザの記事を表示する.ある
イベントなどに関する具体的な情報や,あるトピックに関する専門的な情報を得るこ
とができる.\vspace{1ex}

{\bf (3)} クローズドなユーザ集合に情報を発信するユーザの記事を表示する.
ある内容に関する一般ユーザの反応などといった,個人のつぶやきなどを得る
ことができる.\vspace{1ex}

このように分類することで,自分の求めている情報を効率よく取得することがで
きると考えている.また,\ref{sec:Elements}節で,あるトピックに特化されており,
かつクローズドなユーザ集合に情報を発信するユーザが存在する旨を述べたが,
そのような記事は,上記の(2)と(3)の両方のタブに表示する予定である.