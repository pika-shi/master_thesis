\section{Target Specificity of Twitter Users}
\label{sec:Target Specificity}

In this chapter, we discuss the target specificity of Twitter users, the
measure of to what degree target scope of their information publishing
is specified, and define it.  Then we also discuss why target scope of
their information publishing is specified.

\subsection{Definition of Target Specificity}
\label{subsec:Definition}

In this study, we consider target specificity of Twitter users, as
the measure of to what degree target scope of their information publishing is
specified.  More formally, we define \emph{target specificity} of a
Twitter user as to what degree the user set supposed to be included in
target scope of his information publishing is inclining to a
part of all Twitter users, i.e., to what extent this user set deviates
from the user set randomly sampled from all Twitter users.  In this
paper, we express target specificity of the Twitter user $u$ as
$\mathit{TargetSpecificity}(u)$. This formula takes a range of $[0, 1]$.

For example, a user mainly publishing technical information about
programming is supposed to publish information to programmers.  So
users who are interested in this information are inclining toward a part
of Twitter users.  Thus, it is supposed that target specificity of
the user is high.

On the other hand, a user publishing information about world news
publishes information useful for the public widely.  So the public is
supposed to be interested in this information, and the deviation between
users who are interested in this information and users randomly sampled
from all Twitter users may be very small.  Thus, it is supposed
that target specificity of the user is low.

As mentioned above, target specificity of Twitter users is defined
as to what extent the user set supposed to be included in target scope
of his information publishing deviates from the user set randomly
sampled from all Twitter users.  Thus, the fact that target
specificity of a user is high does not necessarily coincide with the
fact that there is high similarity between users supposed to
be included in target scope of his information publishing each
other.  For example, users supposed to be included in target scope of
information publishing of a user publishing information about
earthquake in a certain area are probably consistent in the area they live in,
and so his/her target specificity is supposed to be high.  But their
other characteristics, e.g., age, sex, interests, communities they
belong to, and so on, vary from user to user.  Thus it is not be able to
be said that they have high similarity each other.  In other words, even
if there are various types of users in target scope of his
information publishing, we consider that target specificity of the
user is high as long as the majority of users in the target scope are
consistent in at least one attribute.

In this paper, we determine a threshold $\delta$. If target
specificity of a user is higher than $\delta$, we call him a
\emph{target user}, and if lower, we call him a \emph{non target
user}.  More formally, we define them as follows:

\vspace{-1ex}
\[
  \begin{cases}
   u\mbox{ is a } target\mbox{ }user, & \mbox{if}\
   \mathit{TargetSpecificity}(u) > \delta \\
   u\mbox{ is a }non\mbox{ }target\mbox{ }user, & \mbox{otherwise}.
  \end{cases}
\]
\vspace{-2ex}

\subsection{Why Target Specificity is High}
\label{subsec:The Causes}

In this subchapter, we discuss what causes target specificity of a
user, i.e., why target scope of his information publishing is
specified.  As a result of our analysis, this is roughly classified into
two causes: $(1)$ because he publish information specified for certain
topics, and $(2)$ because he publish information to the users
specified extensionally.  We discuss their two causes of the target
specificity in the follow.

\begin{description}
\bf {\item[(1)] Specified topics of information extensionally}
\label{item:Topic}
\end{description}

The first cause of target specificity of a user is because
he publishes information specified to a few topics extensionally,
whether he publishes information to the users specified extensionally or
not.  For example, a user mainly publishing technical
information about programming is supposed to publish information to
unspecified users, but it is considered that target scope of
his information publishing is specified because he specifies the
topic of information, i.e., programming.  Furthermore, a user mainly
publishing information about a certain conference is supposed to publish
information to the users who attend the conference or are interested in
it.  Thus it is considered that target scope of his
information publishing is specified.

The way to specify topics of information is roughly classified into two
cases.  In the first case, a user specifies topics based on demographic
data, e.g., age, settled areas, sex, occupation, career, and so on.  It
is able to be said that a user publishing weather information in a
certain area specifies topics based on demographic data.  In the second
case, a user specifies topics based on psychographic data e.g., taste,
hobby, values, and so on.  It is able to be said that a user
publishing information about cooking specifies topics based on
psychographic data.

\begin{description}
\bf{\item[(2)] Specified target users extensionally}
\label{item:User}
\end{description}

The second cause of target specificity of a user is because
he publishes information specified to some users extensionally
whether he specifies topics of his publishing
information or not.  For example, a user communicating with his friends
publishes various contents of information, but it is considered that
target scope of his information publishing is specified because
he specifies users extensionally, i.e., he publishes
information to the closed users, i.e., his friends.  Furthermore, a user
mainly getting in touch with members of a certain club publishes
information to the closed users specified extensionally, i.e., the club
members.  Thus it is considered that target scope of
his information publishing is specified.

Sometimes, both $(1)$ and $(2)$ simultaneously cause the target specificity of a
user.  For example, a user who publishing information to
the members of the artist's fan club publishes information specified not
only users of his information publishing, i.e., the members of the
artist's fan club, but also the topic of information, i.e., the latest news
about the artist and so on.  Furthermore, it is also true in case of a
user notifying students in a certain university of the news toward
them because he publishes information specified extensionally, i.e.,
students of the university, and the
topics of information, i.e., the news toward them.  In addition, some users use
Twitter for the both purpose of publishing information of a certain
topic and communicating with their friends.  It is able to be said that
such users are also an example of the case that both $(1)$ and $(2)$ simultaneously
cause the target specificity of a Twitter user.