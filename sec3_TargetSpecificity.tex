\section{Target Specificity of Twitter Users}
\label{sec:Target Specificity}

In this chapter, we discuss the target specificity of Twitter users, the
measure of to what degree target scope of their information publishing
is specified, and define it.  Then we also discuss why target scope of
their information publishing is specified.

\subsection{Definition of Target Specificity}
\label{subsec:Definition}

In this study, we consider the target specificity of Twitter users, as
the measure of how much target scope of their information publishing is
specified.  More formally, we define the \emph{target specificity} of a
Twitter user as to what degree the user set considered to be included in
target scope of his/her information publishing is inclining toward a
part of all Twitter users, i.e., to what extent this user set deviates
from the user set randomly sampled from all Twitter users.  In this
paper, we express the target specificity of the Twitter user $u$ as
$\mathit{TargetSpecificity}(u)$. This formula takes a range of $[0, 1]$.

For example, a user who mainly publishes technical information about
programming is supposed to publish information toward programmers.  So
users who are interested in this information are inclining toward a part
of Twitter users.  Thus, it may be considered that target specificity of
the user is high.

On the contrary, a user who publishes information about world news
publishes information useful for the public widely.  So the public is
supposed to be interested in this information, and the deviation between
users who are interested in this information and users randomly sampled
from all Twitter users may be very small.  Thus, it may be considered
that target specificity of the user is low.

As mentioned above, the target specificity of Twitter users is defined
as to what extent this user set deviates from the user set randomly
sampled from all Twitter users.  Thus, the fact that the target
specificity of a user is high does not necessarily coincide with the
fact that there is high similarity between users who are considered to
be included in target scope of his/her information publishing each
other.  For example, users considered to be included in target scope of
information publishing of a user who publishes information about
earthquake in a certain area are consistent in the area they live in,
and so his/her target specificity is supposed to be high.  But their
other characteristics such as age, sex, interests, communities they
belong to, and so on vary from user to user.  Thus it is not be able to
be said that they have high similarity each other.  In other words, even
if there are various types of users in target scope of his/her
information publishing, we consider that the target specificity of the
user is high as long as the majority of users in the target scope are
consistent in at least one attribute.

In this paper, we determine a threshold $\delta$. If the target
specificity of a user is higher than $\delta$, we call him/her a
\emph{target user}, and if lower, we call him/her a \emph{non target
user}.  More formally, we define them as follows:

\vspace{-1ex}
\[
  \begin{cases}
   u\mbox{ is a } target\mbox{ }user, & \mbox{if}\
   \mathit{TargetSpecificity}(u) > \delta \\
   u\mbox{ is a }non\mbox{ }target\mbox{ }user, & \mbox{otherwise}.
  \end{cases}
\]
\vspace{-2ex}

\subsection{Why Target Specificity is High}
\label{subsec:The Causes}

In this subchapter, we discuss what causes the target specificity of a
Twitter user, i.e., why target scope of their information publishing is
specified.  As a result of our analysis, this is roughly classified into
two causes: $(1)$ topics of information and $(2)$ target users.  We
discuss their two causes of the target specificity in the follow.

\begin{description}
\bf {\item[(1)] Specified topics of information extensionally}
\label{item:Topic}
\end{description}

The first cause of the target specificity of a Twitter user is because
he/she publishes information specified to a few topics extensionally
whether or not he/she specifies target users of his/her information
publishing.  For example, a user who mainly publishes technical
information about programming is supposed to publish information to
unspecified people, but it may be considered that target scope of
his/her information publishing is specified because he/she specifies the
topic of information: programming.  Furthermore, a user who mainly
publishes information about a certain conference is supposed to publish
information to the users who attend the conference or are interested in
it.  Because of this, it may be considered that target scope of his/her
information publishing is specified.

The way to specify topics of information is roughly classified into two
cases.  In the first case, a user specifies topics based on demographic
data such as age, settled areas, sex, occupation, career, and so on.  It
is able to be said that a user who publishes weather information in a
certain area specifies topics based on demographic data.  In the second
case, a user specifies topics based on psychographic data such as taste,
hobby, values, and so on.  It is able to be said that a user who
publishes information about cooking specifies topics based on
psychographic data.

\begin{description}
\bf{\item[(2)] Specified target users extensionally}
\label{item:User}
\end{description}

The second cause of the target specificity of a Twitter user is because
he/she publishes information specified to some users extensionally
whether or not he/she specifies topics of his/her publishing
information.  For example, a user who communicates with his/her friends
publishes various contents of information, but it may be considered that
target scope of his/her information publishing is specified because
he/she specifies target users extensionally, i.e., he/she publishes
information to the closed users: his/her friends.  Furthermore, a user
who mainly gets in touch with members of a certain club publishes
information to the closed users specified extensionally: the club
members.  Because of this, it may be considered that target scope of
his/her information publishing is specified.

Sometimes, both $(1)$ and $(2)$ cause the target specificity of a Twitter
user simultaneously.  For example, a user who publishes information to
the members of the artist's fan club publishes information specified not
only target users of his/her information publishing: the members of
artist's fan club, but also the topic of information: the latest news
about the artist and so on.  Furthermore, it is also true in case of the
user who notifies students of a certain university of the news toward
them because he/she publishes information specified target users of
his/her information publishing: students of the university, and the
topics of information the news toward them.  And also, some users use
Twitter for the both purpose of publication information of a certain
topic and communication with their friends.  It is able to be said that
such users are also an example of the case that both $(1)$ and $(2)$
cause the target specificity of a Twitter user simultaneously.